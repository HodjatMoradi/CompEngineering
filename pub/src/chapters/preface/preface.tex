%%
%% Automatically generated file from DocOnce source
%% (https://github.com/hplgit/doconce/)
%%

% #define PREAMBLE

% #ifdef PREAMBLE
%-------------------- begin preamble ----------------------

\documentclass[%
oneside,                 % oneside: electronic viewing, twoside: printing
final,                   % draft: marks overfull hboxes, figures with paths
10pt]{article}

\listfiles               %  print all files needed to compile this document

\usepackage{relsize,makeidx,color,setspace,amsmath,amsfonts,amssymb}
\usepackage[table]{xcolor}
\usepackage{bm,ltablex,microtype}

\usepackage[pdftex]{graphicx}

\usepackage[T1]{fontenc}
%\usepackage[latin1]{inputenc}
\usepackage{ucs}
\usepackage[utf8x]{inputenc}

\usepackage{lmodern}         % Latin Modern fonts derived from Computer Modern

% Hyperlinks in PDF:
\definecolor{linkcolor}{rgb}{0,0,0.4}
\usepackage{hyperref}
\hypersetup{
    breaklinks=true,
    colorlinks=true,
    linkcolor=linkcolor,
    urlcolor=linkcolor,
    citecolor=black,
    filecolor=black,
    %filecolor=blue,
    pdfmenubar=true,
    pdftoolbar=true,
    bookmarksdepth=3   % Uncomment (and tweak) for PDF bookmarks with more levels than the TOC
    }
%\hyperbaseurl{}   % hyperlinks are relative to this root

\setcounter{tocdepth}{2}  % levels in table of contents

% prevent orhpans and widows
\clubpenalty = 10000
\widowpenalty = 10000

% --- end of standard preamble for documents ---


% insert custom LaTeX commands...

\raggedbottom
\makeindex
\usepackage[totoc]{idxlayout}   % for index in the toc
\usepackage[nottoc]{tocbibind}  % for references/bibliography in the toc

%-------------------- end preamble ----------------------

\begin{document}

% matching end for #ifdef PREAMBLE
% #endif

\newcommand{\exercisesection}[1]{\subsection*{#1}}

make.sh: unsuccessful command preprocess -DFORMAT=pdflatex ../newcommands.p.tex
abort!


% ------------------- main content ----------------------

What does computers do better than humans? What is it possible to
compute? These questions have not been fully answered yet, and in the
coming years we will most likely see that the boundaries for what
computers can do will expand significantly. Many of the  fundamental laws in
nature have been known for quite some time, but still it is almost
impossible to predict the behavior of water (H$_2$O) from quantum
mechanics. The most sophisticated super computers runs for days and are
only able to simulate  the behavior of molecules in a couple of
seconds, almost too short to extract meaningful thermodynamic
properties. This leads to another interesting question: What does humans do better
than machines? A large part of
the answer to this question is \emph{modeling}. Modeling is the ability to 
break a complicated, unstructured problem into smaller pieces that can
be solved by computers or by other means. Modeling requires \emph{domain knowledge}, one need to
understand the system well enough to make the correct or the most efficient simplifications. The process usually starts
with some experimental data that one would like to understand, it could be the increasing temperature in the atmosphere or sea, it could 
be changes in the chemical composition of a fluid passing through a rock. The modeler then makes a mental image, which includes a set of 
mechanisms that could be the cause of the observed data. These mechanisms then needs to be formulated mathematically.   
How can we know if a model of a system is good? First of all, a good model is a model that do not break 
any of the fundamental laws of nature, such as mass (assuming non relativistic effects) and energy conservation. Even if you are searching 
for new laws of nature, you have to make sure that your model respect the existing laws, because then a deviation from your model and
the observations could be a hint of the new physics you are searching for.  
Secondly, the model must be able to match the observable data, with a limited set of variables. The variables should 
be determined from data, and then the model should be able to make some predictions that can be tested. Thus, the true
purpose of the model is not only to match experimental data, but serve as a framework where the underlying
mechanisms of the process can be understood. This is done by making model predictions, test them, and improve the model.

In this course our main focus will be on how to use computers to solve models. We will show you through exercises how a mathematical model of
a physical system can be made, and you will have the possibility to explore the model. Computers are extremely useful, they can solve problems that
would be impossible to solve by hand. However, it is extremely important to know about the limitations and strength of various algorithms. One need
to have a toolbox of various algorithms that can be employed depending on the problem one are studying. Sometimes speed is not an issue, and one can use
simpler algorithms, but in many cases \emph{speed is an issue}. Thus it is important to not waste computational time when it is not needed, we will encounter 
examples of this many times in this course. Why should you spend time learning about algorithms that have been implemented already in a software that 
most likely can be downloaded for free? There are many answers to this question, some more practical and some that goes deeper. Lets start with the
practical considerations: Often you encounter a problem that needs to be solved by a computer, it could be as simple as to integrate some production data 
in a spreadsheet to calculate the total production, or it could be to fit a function with more than one variable to some data. Once you have this problem, and 
starting to ask Mr.~Google for a solution, you will quickly realize that there are numerous ways of achieving what you want. By educating yourself 
within the most basic numerical methods, presented in this course, you will be able to judge for yourself which method to use in a specific case. 
Another motivation is that development of most of the different numerical methods are \emph{not that difficult}, they usually follow a very similar pattern, but
there are some ''tricks''. It is extremely useful to learn these tricks, they can be adopted to a range of different problems, many are easily implemented
in a spreadsheet. There are some more deeper arguments, and that is that the numerical methods are developed to solve a \emph{general} problem. Most of the 
time we work with \emph{specific} problems, and we would like to have an algorithm that is optimal for our problem that goes beyond only choosing the right one. 
Having understood and learned all the cool tricks that was used in the development of the algorithm in the general case, 
is a starting point for adopting the algorithm to your specific situation. Secondly development of an algorithm is a concrete case of \emph{Computational Thinking}.
Computational thinking is not necessarily related to computers and programming, but it is a way of structuring your work 
into precise statements that are being executed one at a time in a specific order. By learning about algorithmic development, you 
will train yourself in the art of computational thinking, which is a useful skill in all kind of problem solving. 


\noindent
{\it November 2018}  \hfill  {\it Aksel Hiorth}

% ------------------- end of main content ---------------

% #ifdef PREAMBLE
\end{document}
% #endif

